\documentclass[a4paper]{article}

% Importazioni
\usepackage{import}
% ----- Codifica, lingua, font -----
\usepackage[T1]{fontenc}
\usepackage[utf8]{inputenc}
\usepackage[italian]{babel}

% ----- Layout e impaginazione -----
\usepackage{geometry}
\geometry{a4paper, top=3cm, bottom=3cm, left=3cm, right=3cm}

\usepackage{titlesec}
\usepackage{fancyhdr}
\usepackage{graphicx}
\usepackage{float}
\usepackage{caption}
\usepackage{subcaption}
\setlength{\parindent}{0pt}

\usepackage{booktabs}
\usepackage{csquotes}
\usepackage[table]{xcolor}

% ----- Numerazione sezioni fino al 4° livello -----
\setcounter{secnumdepth}{5}
\setcounter{tocdepth}{5}

% ----- Sezione fino al 4° livello (subsubsubsection) -----
\newcounter{subsubsubsection}[subsubsection]
\renewcommand{\thesubsubsubsection}{\thesubsubsection.\arabic{subsubsubsection}}
\newcommand{\subsubsubsection}[1]{%
    \refstepcounter{subsubsubsection}%
    \addcontentsline{toc}{subparagraph}{\thesubsubsubsection\quad #1}%
    \vspace{1ex}\noindent\textbf{\thesubsubsubsection\quad #1}\par\nopagebreak\vspace{0.5ex}
}

% ----- Intestazione/Piè di pagina -----
\pagestyle{fancy}
\fancyhead[L]{\nouppercase{\leftmark}}
\fancyhead[R]{\nouppercase{\rightmark}}
\fancyfoot[C]{\thepage}

% ----- Stile titoli sezione -----
\titleformat{\section}
{\normalfont\Large\bfseries\centering}
{\makebox[0pt][r]{\textcolor{gray}{\fontsize{40}{40}\selectfont\thesection\hspace{0.3em}}}}
{0pt}
{\Huge}
[\vspace{1em}\titlerule]

\titlespacing*{\section}{0pt}{45pt}{20pt}

% Interruzione automatica prima di ogni sezione
\let\oldsection\section
\renewcommand\section{\newpage\oldsection}

% ----- Colori personalizzati -----
\definecolor{verde}{RGB}{0, 153, 0}
\arrayrulecolor[HTML]{808080}

% ----- Tipografia matematica -----
\usepackage{amsmath, amssymb, amsthm, mathtools}
\usepackage{mdframed}
\newmdtheoremenv{theo}{Teorema}

% ----- Codice e pseudocodice -----
\usepackage{listings}
\lstdefinelanguage{Pseudocode}{
    keywordstyle=\color{blue}\bfseries,
    keywords={if, else, while, for, function, return, end},
    ndkeywords={input, output, var},
    sensitive=false,
    comment=[l]{//},
    morecomment=[s]{/*}{*/},
    morestring=[b]",
    basicstyle=\ttfamily,
    stringstyle=\color{red},
    commentstyle=\color{green!60!black},
}
\lstdefinestyle{modern}{
    backgroundcolor=\color{blue!5},
    commentstyle=\color{gray!70},
    keywordstyle=\color{blue}\bfseries,
    stringstyle=\color{green!70},
    basicstyle=\ttfamily\small,
    numberstyle=\tiny\color{blue},
    numbers=left,
    stepnumber=1,
    frame=single,
    framesep=3pt,
    frameround=tttt,
    rulecolor=\color{blue!35},
    breakatwhitespace=false,
    tabsize=2,
    captionpos=b,
    escapeinside={(*@}{@*)},
    mathescape=true,
}

% ----- Grafici e figure -----
\usepackage{tikz}
\usetikzlibrary{shapes.misc}
\usepackage{pgfplots}
\pgfplotsset{compat=1.18}
\usepgfplotslibrary{fillbetween}
\usepackage{pgf-pie}

% ----- Citazioni, bibliografia, note -----
\usepackage{quoting}
\usepackage[
    backend=biber,
    style=alphabetic,
    sorting=ynt
]{biblatex}
\addbibresource{bibliografia.bib}

% ----- Hyperlink (nascosti ma attivi) -----
% Carica hyperref con le opzioni per i link interni (document links)
\usepackage{hyperref}
\hypersetup{
    colorlinks=true,  % Rende i link colorati anziché avere un bordo
    linkcolor=blue,   % Colore per i link interni (es. \ref, \autoref, \hyperlink)
    citecolor=green,  % Colore per le citazioni
    filecolor=magenta,
    urlcolor=cyan,
    pdftitle={Your Document Title}, % Aggiungi un titolo per il PDF
    pdfauthor={Your Name}, % Aggiungi l'autore del PDF
    pdfsubject={Your Subject}, % Aggiungi l'argomento del PDF
    pdfkeywords={Keyword1, Keyword2}, % Aggiungi parole chiave
}

% Ridefinisce \tableofcontents per avere link invisibili
\let\oldtableofcontents\tableofcontents
\renewcommand{\tableofcontents}{%
    \begingroup
    \hypersetup{linkcolor=black, hidelinks=true} % Nasconde i link solo per l'indice
    \oldtableofcontents
    \endgroup
}


% ----- Titolo con sottotitolo -----
\usepackage[cc]{titlepic}
\makeatletter
\providecommand{\subtitle}[1]{%
    \apptocmd{\@title}{\par {\large #1 \par}}{}{}
}
\makeatother

% ----- Spaziatura tabelle -----
\setlength{\arrayrulewidth}{0.3mm}
\setlength{\tabcolsep}{18pt}
\renewcommand{\arraystretch}{1.5}
\newcolumntype{s}{>{\columncolor[HTML]{FFFFFF}} p{3cm}}

% ----- Comandi personalizzati -----
\newcommand{\lapl}{\mathcal{L}}
\colorlet{stepnote}{black}
\newcommand{\stepnote}[1]{&&\text{\color{stepnote}#1}}

\begin{document}

% Titolo
\begin{titlepage}
    $ $
    \vspace{5pt}
    \newcommand{\HRule}{\rule{\linewidth}{0.5mm}}
    \begin{center}
    \textsc{\normalsize track tune}\\[1cm]
    %\text{\large Relazione conclusiva}\\[0.5cm] % Minor heading such as course title
    \end{center}
    
    \center
     
    %----------------------------------------------------------------------------------------
    %	HEADING SECTIONS
    %----------------------------------------------------------------------------------------
    
    \HRule \\[0.4cm]
    { \huge \bfseries SW Engineering Documentation}\\[0.4cm] % Title of your document
    \HRule \\[1.5cm]
     
    %----------------------------------------------------------------------------------------
    %	AUTHOR SECTION
    %----------------------------------------------------------------------------------------
    
    \begin{minipage}{0.4\textwidth}
    \begin{flushleft} \large
    \emph{Autori:}\\
    Mattia \textsc{Rebonato} \\
    Davide \textsc{Fozzato}\\
    \end{flushleft}
    \end{minipage}
    ~
    \begin{minipage}{0.4\textwidth}
    \begin{flushright} \large
    \emph{Professore:} \\
    Prof Carlo \textsc{Combi}
    \end{flushright}
    \end{minipage}\\[2cm]
    
    \vspace{250pt}
    
    \textsc{\normalsize A.A. 2024-2025}\\[0.5cm]
    
    \vfill % Fill the rest of the page with whitespace
    \end{titlepage}

\tableofcontents
\pagebreak

% =======================
\section{Introduzione}
% =======================

\subsection{Prefazione}

Il presente documento definisce in modo esaustivo i requisiti, l'architettura e 
le specifiche di progettazione del software \textit{TrackTune}, 
una piattaforma applicativa dedicata alla gestione, condivisione e analisi 
collaborativa di contenuti musicali.\\[2ex] 

La documentazione è rivolta a sviluppatori, tester, amministratori di sistema e a 
tutti gli stakeholder coinvolti nella realizzazione, manutenzione e gestione del 
progetto.

\subsection{Panoramica}

La piattaforma si rivolge a musicisti, compositori, insegnanti, studenti e appassionati
di musica, offrendo un ambiente digitale comodo per l’archiviazione, la collaborazione 
e l’accesso a materiale musicale di vario genere, tra cui spartiti, testi, accordi, 
file audio/video e link esterni (ad esempio YouTube).

Essa si propone non solo come archivio digitale, 
ma anche come spazio per l’approfondimento culturale, 
lo scambio di interpretazioni e feedback tra utenti, favorendo la condivisione 
di esperienze musicali, commenti critici e annotazioni esecutive su brani e 
contenuti attinenti alla musica.

% ==============================
\section{Analisi dei requisiti}
% ==============================

\subsection{Descrizione del sistema}

Si vuole progettare un sistema software per gestire la collezione e la condivisione 
di spartiti, testi, accordi, MIDI, MP3, video, link e molto altro, relativamente a brani 
musicali di diversa tipologia/genere.\\[2ex]

Gli utenti, previa autorizzazione dell'amministratore, devono poter caricare, 
scaricare, commentare e interagire con i vari contenuti. Per ogni risorsa il sistema deve
consentire di specificare:
\begin{itemize}
    \item autori
    \item genere/generi
    \item strumenti musicali utilizzati
\end{itemize}

Inoltre, per le risorse multimediali (audio, video, ecc.) il sistema deve
consentire di specificare il luogo e la data di registrazione.

Ogni utente può aggiungere note di testo libero su segmenti specifici di esecuzione (MP3, MP4 o video YouTube).  
Un segmento è definito dal momento di inizio e fine, e per ciascuno è possibile includere dettagli come assoli, esecutori, strumenti,
ritmi e altre caratteristiche. I commenti possono essere arricchiti con ulteriori 
risposte, consentendo una profondità illimitata di discussioni relativamente a quel 
contenuto.\\[2ex]

Un brano può essere inserito anche da un autore o un interprete. Deve essere possibile 
quindi distinguere il proprio ruolo all’interno del brano e, nel caso di interpreti, 
gli strumenti utilizzati. I commenti sulle modalità esecutive devono apparire in maniera 
rilevante rispetto a quelli di altri utenti.\\[2ex]

I video YouTube possono essere visualizzati direttamente nel software o nel browser, 
ma tutti i commenti relativi sono gestiti dal sistema.\\[2ex]

Gli utenti possono aggiungere commenti su spartiti, testi e accordi, 
relativi alle modalità esecutive (come strumenti, ritmo, intensità, ecc.), sia su 
specifiche parti del brano che sull'intero brano.\\[2ex]

Un utente può richiedere la registrazione alla piattaforma tramite un form, che 
l'amministratore valuterà e risponderà a seconda dell’esito.  
L'amministratore gestisce gli utenti, ha il compito di rimuovere coloro che pubblicano contenuti non pertinenti 
e di moderare i commenti. Inoltre, valuta le richieste di registrazione, il cui esito verrà comunicato 
all’utente interessato.\\[2ex]

Il sistema deve permettere la ricerca delle risorse tramite i seguenti filtri:
\begin{itemize}
    \item genere
    \item titolo del brano
    \item autore/i
\end{itemize}

Ogni risorsa per cui un utente ha lasciato un commento deve essere direttamente accessibile.

Gli strumenti musicali, i generi, i titoli dei brani e i nomi degli autori vanno gestiti attraverso dizionari aggiornabili, da usare anche per le opportune ricerche.

\subsection{Requisiti funzionali}

\subsubsection{Gestione degli utenti}
\begin{itemize}
    \item Registrazione tramite form dedicato;
    \item Valutazione da parte dell’amministratore delle richieste di registrazione e relativa comunicazione dell’esito;
    \item Profilazione (distinzione tra utente/amministratore);
    \item Autorizzazione al caricamento dei file per utente;
\end{itemize}

\subsubsection{Gestione dei contenuti}
\begin{itemize}
    \item Aggiunta, modifica e rimozione di contenuti multimediali
    \begin{itemize}
        \item La rimozione può essere eseguita solo dall’amministratore o dall’utente che ha caricato il contenuto
    \end{itemize}
    \item Visualizzazione delle risorse tramite software di terze parti o tramite l’applicativo stesso.
    \item Moderazione degli utenti
    \begin{itemize}
        \item L’amministratore può rimuovere utenti;
        \item L’amministratore può rimuovere commenti/risorse inappropriate.
    \end{itemize}
    \item Gestione dei dizionari aggiornabili (solo amministatore) relativi a:
    \begin{itemize}
        \item Autori;
        \item Generi;
        \item Strumenti musicali;
    \end{itemize}
\end{itemize}

\subsubsection{Gestione dei commenti}
\begin{itemize}
    \item Possibilità di commentare brani;
    \item Possibilità di rispondere ai commenti;
    \item Possibilità di commentare su specifici segmenti di esecuzione;
    \item Visualizzazione differenziata dei commenti per autori/esecutori del brano.
\end{itemize}

\subsection{Casi d'uso}
L’applicazione sarà disponibile esclusivamente agli utenti abilitati dall’amministratore, il quale è anch’esso un utente ma con privilegi di amministrazione sui comportamenti del sistema. L’unica operazione che può effettuare un utente non registrato è la richiesta di creazione di un account.
\end{document}
